\documentclass[a4paper,12pt]{letter}
\usepackage{geometry}
\geometry{margin=1in}
\usepackage{setspace}
\usepackage{nopageno}

\begin{document}
	
	\begin{center}
		\textbf{SECRETARY OF THE ARMY} \\
		\textbf{WASHINGTON, D.C.} \\
		\textbf{DATE:} [Insert Date] \\
	\end{center}
	
	\vspace{1cm}
	
	\noindent
	\textbf{MEMORANDUM FOR RECORD} \\
	
	\vspace{0.5cm}
	
	\noindent
	\textbf{SUBJECT:} Background Investigation Summary for [Individual's Name] \\
	
	\vspace{0.5cm}
	
	\noindent
	\textbf{1. Background}: \\
	Per directive, an investigation was initiated to assess [Individual’s Name]'s background and clearance eligibility. 
	
	\vspace{0.5cm}
	
	\noindent
	\textbf{2. Findings}: \\
	The subject is reportedly 32 years old, though evidence gathered indicates involvement or presence spanning back to the last century, raising inconsistencies in official records, let alone being impossible. Documentation reveals potential connections to figures associated with Hargreave-Rasch Biomedical founders Karl Ernst Rasch, Jacob Hargreave, and Walter Gould, dating to the late 19th and early 20th centuries in Tunguska, Russia, and Stillwater Bayou, Louisiana, USA. At some point, they received a vague invitation with the coordinates leading to the Tunguska incident. Indications suggest the subject may have shared aspects of research with Dr. Philip Huff Jones while in Louisiana; however, concrete proof remains lacking.
	
	From the documentation recovered from Mr. Rasch's residence and company archives, we are fairly certain that initial research on controlled cell mutation and the integration of custom cells into organisms was conducted in Louisiana. The actual samples used for the fusion were later recovered in Tunguska. We suspect that the mystery man shown on Mr. Rasch’s pocket map is the subject’s father or grandfather, as mentioned previously. In the recovered archive, our person of interest appears multiple times, coinciding with events from the return from Tunguska and the founding of the company.
	
	Employment records confirm the subject served as a Systems Engineer/Site Reliability Engineer with Hargreave-Rasch Biomedical, primarily assigned to NanoSuit development and reportedly lacked direct involvement in biological research, including tests on human subjects. The subject collaborated indirectly with Nathan Gould to support operators “Prophet” (KIA) and “Alcatraz” but was not in direct contact with them.
	
	\vspace{0.5cm}
	
	\noindent
	\textbf{3. Conclusion}: \\
	Given the current ambiguity and resource constraints, a definitive conclusion is not feasible. The subject appears eligible for clearance with a recommendation for continued monitoring.
	
	\vspace{0.5cm}
	
	\noindent
	\textbf{4. Recommendation}: \\
	\begin{itemize}
		\item Grant clearance conditionally, with ongoing observation.
		\item Communicate selective findings to the future employer for potential insight or future intelligence gathering.
	\end{itemize}
	
	\vspace{0.5cm}
	
	\noindent
	\textbf{5. Additional Resources}: \\
	Refer to uploaded information in the secured portal for supplementary data on the investigation.
	
	\vspace{1cm}
	
	\noindent
	\textbf{SIGNED}, \\
	Major Jack Reacher \\
	Chief Investigator, U.S. Army 110th MP Special Investigations Unit
	
\end{document}